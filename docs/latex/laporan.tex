\documentclass{article}

\usepackage[parfill]{parskip}
\usepackage{
  geometry,
  float,
  listings,
  xcolor,
  amssymb,
  fontspec,
  graphicx,
  hyperref,
}

\graphicspath{ {../imgs/} }

% To keep figures, listings, and tables rendered where they're put
% *** BEGIN ***
\let\origfigure\figure
\let\endorigfigure\endfigure
\renewenvironment{figure}[1][2] {
    \expandafter\origfigure\expandafter[H]
} {
    \endorigfigure
}

\AtBeginDocument{\floatplacement{codelisting}{H}}
\AtBeginDocument{\floatplacement{table}{H}}
% *** END ***

\geometry{
  top = 20mm,
  bottom = 20mm,
  left = 25mm,
  right = 25mm,
  paper = a4paper,
}

\setmainfont{Times New Roman}
\setmonofont{JetBrainsMono Nerd Font Mono}

\definecolor{codegreen}{rgb}{0,0.6,0}
\definecolor{codegray}{rgb}{0.5,0.5,0.5}
\definecolor{codepurple}{rgb}{0.58,0,0.82}
\definecolor{backcolour}{rgb}{0.95,0.95,0.92}

\lstset{% https://en.wikibooks.org/wiki/LaTeX/Source_Code_Listings
  basicstyle = \footnotesize\ttfamily,
  breakatwhitespace = true,
  breaklines = true,
  frame = single,
  firstnumber = 1,
  keepspaces = true,
  numbers = left,
  tabsize = 4,
  backgroundcolor=\color{backcolour},
  commentstyle=\color{codegray},
  keywordstyle=\color{orange},
  numberstyle=\tiny\color{codegray},
  stringstyle=\color{codegreen},
  basicstyle=\ttfamily\footnotesize,
}

% https://github.com/gilangardya/if2211-cryptarithmetic-solver/blob/master/doc/LaporanTucil1.pdf

\begin{document}
\begin{titlepage}
  \centering
  \vspace*{\stretch{2}}
  \Large Laporan Tugas Kecil I

  \large Dekripsi \textit{Cryptarithmetic} dengan \textit{Brute Force}

  \normalsize

  \vspace{\stretch{1}}

  \includegraphics[scale=0.2]{logo-itb.png}

  \vspace{\stretch{1}}
  \begin{tabular}{lll}
    Nama  &: & Josep Marcello \\
    NIM &: & 13519164 \\
    Kelas &: & K-03 \\
    Dosen &: & Prof.\ Dwi Hendratmo Widyantoro \\
    Bahasa &: & Java \\
  \end{tabular}

  \vspace{\stretch{2}}
  \large
  PROGRAM STUDI TEKNIK INFORMATIKA

  SEKOLAH TEKNIK ELEKTRO DAN INFORMATIKA

  INSTITUT TEKNOLOGI BANDUNG

  BANDUNG

  2021

  \vspace{\stretch{2}}
\end{titlepage}

\section{Algoritma \textit{Decrease and Conquer}}
\begin{enumerate}
\end{enumerate}

\section{\textit{Source Code} Program}
\begin{lstlisting}[caption = main.java, language = java]
\end{lstlisting}

\section{Hasil Pengujian}

\subsection{Tabel Penilaian}
\begin{table}
  \begin{center}
    \begin{tabular}{|p{7cm} | l | l|}
      \hline
      Poin & Ya & Tidak \\
      \hline
      1. Program berhasil dikompilasi & \checkmark & \\
      \hline
      2. Program berhasil \textit{running} & \checkmark & \\
      \hline
      3. Program dapat menerima berkas input dan menuliskan output & \checkmark & \\
      \hline
      4. Luaran sudah benar untuk semua kasus input & & \checkmark \\
      \hline
    \end{tabular}
  \end{center}
\end{table}

\section*{\textit{Link} ke \textit{repository} Github}
\href{https://github.com/jspmarc/Tucil1_13519164}{\textit{Link} ke
\textit{repository}}: https://github.com/jspmarc/Tucil1\_13519164.

\end{document}
